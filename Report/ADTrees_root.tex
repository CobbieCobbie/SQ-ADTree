\documentclass[conference]{IEEEtran}
\IEEEoverridecommandlockouts
% The preceding line is only needed to identify funding in the first footnote. If that is unneeded, please comment it out.
\usepackage{cite}
\usepackage{amsmath,amssymb,amsfonts}
\usepackage{algorithmic}
\usepackage{graphicx}
\usepackage{textcomp}
\usepackage{xcolor}
\usepackage{verbatim}
\def\BibTeX{{\rm B\kern-.05em{\sc i\kern-.025em b}\kern-.08em
    T\kern-.1667em\lower.7ex\hbox{E}\kern-.125emX}}

%% Bibliography1

\begin{document}

\title{Attack-Defense Trees}

\author{\IEEEauthorblockN{Benjamin \c Coban}
\IEEEauthorblockA{\textit{Eberhard Karls University Tübingen} \\
\textit{Embedded Systems}\\
Tübingen, D \\
Benjamin.Coban@student.uni-tuebingen.de}
}

\maketitle

\begin{abstract}
This document is a model and instructions for \LaTeX.
This and the IEEEtran.cls file define the components of your paper [title, text, heads, etc.]. *CRITICAL: Do Not Use Symbols, Special Characters, Footnotes, 
or Math in Paper Title or Abstract.
\end{abstract}

\begin{IEEEkeywords}
Attack Trees, ADTerms, Attribute Domains
\end{IEEEkeywords}

\section{Introduction}
\section{Terminology}
A \textit{graph} $G=(V_G,E_G)$ is a tuple consisting of two sets - the set of vertices and the set of edges. An \textit{edge} $e = (v,w), v,w \in V_G$ is a tuple and describes a connectivity relation between two vertices. Unless otherwise mentioned, the graphs are \textit{undirected}, meaning that the edge $(u,v)$ is identical to the edge $(v,u)$,$u,v\in V_G$. A \textit{Tree} $T$ 
\section{Abstract syntax: ADTerms}
\section{Propositional semantics}
\section{Attributes}
\section{Summary}
\section{Future Work}
\begin{comment}
	Please number citations consecutively within brackets \cite{b1}. The 
	sentence punctuation follows the bracket \cite{b2}. Refer simply to the reference 
	number, as in \cite{b3}---do not use ``Ref. \cite{b3}'' or ``reference \cite{b3}'' except at 
	the beginning of a sentence: ``Reference \cite{b3} was the first $\ldots$''
	
	Number footnotes separately in superscripts. Place the actual footnote at 
	the bottom of the column in which it was cited. Do not put footnotes in the 
	abstract or reference list. Use letters for table footnotes.
	
	Unless there are six authors or more give all authors' names; do not use 
	``et al.''. Papers that have not been published, even if they have been 
	submitted for publication, should be cited as ``unpublished'' \cite{b4}. Papers 
	that have been accepted for publication should be cited as ``in press'' \cite{b5}. 
	Capitalize only the first word in a paper title, except for proper nouns and 
	element symbols.
	
	For papers published in translation journals, please give the English 
	citation first, followed by the original foreign-language citation \cite{b6}.
\end{comment}
\begin{thebibliography}{00}
\bibitem{origin} G. Eason, B. Noble, and I. N. Sneddon, ``On certain integrals of Lipschitz-Hankel type involving products of Bessel functions,'' Phil. Trans. Roy. Soc. London, vol. A247, pp. 529--551, April 1955.
\end{thebibliography}
\end{document}
